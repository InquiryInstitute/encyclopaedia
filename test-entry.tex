\documentclass{encyclopaedia}

% Volume metadata for running headers
\renewcommand{\volumenum}{I}
\renewcommand{\volumetitle}{Mind}

\title{The Encyclopædia}
\renewcommand{\subtitle}{Volume I: Mind}
\author{The Inquiry Institute}
\date{2026}

\begin{document}

% Title page
\maketitle

% Table of contents
\tableofcontents

\cleardoublepage

% Start two-column layout for entries
\twocolumn

% Test entry
\entry{ATTENTION}

Attention is the selective focus of consciousness on particular objects, thoughts, or experiences while excluding others. This fundamental capacity of mind has been examined from multiple perspectives throughout the history of inquiry.

From a psychological standpoint, attention involves the allocation of limited cognitive resources. William James, in his \textit{Principles of Psychology}, described attention as ``the taking possession by the mind, in clear and vivid form, of one out of what seem several simultaneously possible objects or trains of thought.'' This definition captures both the selective nature of attention and its role in consciousness.

The economic and computational perspective, advanced by Herbert A. Simon, frames attention as a scarce resource that must be allocated efficiently. In an information-rich world, attention becomes the limiting factor, not information itself. This view has profound implications for understanding human decision-making and the design of systems that compete for human attention.

Attention operates at multiple levels: voluntary attention, where we deliberately direct focus; involuntary attention, where salient stimuli capture our awareness; and sustained attention, the ability to maintain focus over time. Each of these modes involves different neural mechanisms and can be disrupted in various ways.

The relationship between attention and consciousness remains a subject of active inquiry. Some argue that attention is necessary for consciousness, while others suggest that attention and consciousness can be dissociated. The study of attention disorders, such as attention deficit hyperactivity disorder, provides clinical insights into the mechanisms and importance of this capacity.

\marginalia{William James}{Comment (1890)}{Attention is not a faculty but a function—the function of selection.}

\marginalia{Herbert A. Simon}{Objection (1971)}{Attention is constrained not by will but by computational limits.}

\authorsignature{William James}{}

\clearpage

% Second test entry
\entry{CONSCIOUSNESS}

Consciousness presents one of the most persistent puzzles in philosophy and science. What is it? How does it arise? Why does it exist at all?

The phenomenological tradition, beginning with Edmund Husserl, approaches consciousness through careful description of experience itself. Consciousness is always consciousness \textit{of} something—it is intentional, directed toward objects. This intentional structure reveals consciousness not as a thing but as a process, a way of being in the world.

From a biological perspective, consciousness appears to be an emergent property of complex neural systems. Charles Darwin's evolutionary framework suggests that consciousness, like other biological traits, must confer some adaptive advantage. Yet the precise mechanisms by which neural activity gives rise to subjective experience—the so-called ``hard problem'' of consciousness—remains unresolved.

Computational approaches, following Alan Turing, explore whether consciousness might be understood as a form of information processing. Could a sufficiently complex computational system be conscious? This question challenges our intuitions about what consciousness is and what it requires.

Mystical traditions, represented here by Meister Eckhart, offer yet another perspective: consciousness may be fundamentally unified, and the sense of individual consciousness may be an illusion or limitation. The apophatic approach—speaking of consciousness through negation—suggests that consciousness may be ineffable, beyond the reach of conceptual thought.

These perspectives need not be mutually exclusive. Consciousness may be intentional \textit{and} biological \textit{and} computational \textit{and} ultimately mysterious. The Encyclopædia presents these views without synthesis, allowing each to stand on its own terms.

\marginalia{Edmund Husserl}{Extension (1913)}{Consciousness is not a container but a field of intentional relations.}

\marginalia{Alan Turing}{Question (1950)}{If a machine can think, can it be conscious?}

\authorsignature{Edmund Husserl}{}

\end{document}
